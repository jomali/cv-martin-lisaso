%%%%%%%%%%%%%%%%%%%%%%%%%%%%%%%%%%%%%%%%%
% Developer CV
% LaTeX Template
% Version 1.0 (28/1/19)
%
% This template originates from:
% http://www.LaTeXTemplates.com
%
% Authors:
% Jan Vorisek (jan@vorisek.me)
% Based on a template by Jan Küster (info@jankuester.com)
% Modified for LaTeX Templates by Vel (vel@LaTeXTemplates.com)
%
% License:
% The MIT License (see included LICENSE file)
%
%%%%%%%%%%%%%%%%%%%%%%%%%%%%%%%%%%%%%%%%%

%
%	PACKAGES AND OTHER DOCUMENT CONFIGURATIONS
%

\documentclass[9pt]{developercv} % Default font size, values from 8-12pt are recommended

\begin{document}

%
%	TITLE AND CONTACT INFORMATION
%

\begin{minipage}[t]{0.6\textwidth}
	\vspace{-\baselineskip} % Required for vertically aligning minipages

	\colorbox{black}{{\HUGE\textcolor{white}{\textbf{\MakeUppercase{Martín Lisaso,}}}}}
	\colorbox{black}{{\HUGE\textcolor{white}{\textbf{José Francisco}}}}

	\vspace{6pt}

	{\huge Desarrollador \emph{front-end}}
\end{minipage}
\hfill
\begin{minipage}[t]{0.3\textwidth}
	\vspace{-\baselineskip} % Required for vertically aligning minipages

	\icon{MapMarker}{12}{Santander, Cantabria}\\
	\icon{Phone}{12}{+34 651 667 661}\\
	\icon{At}{12}{\href{mailto:jfm.lisaso@gmx.com}{jfm.lisaso@gmx.com}}\\
	\icon{Globe}{12}{\href{https://jomali.itch.io/}{jomali.itch.io}}\\
	% \icon{Github}{12}{\href{https://github.com/jomali}{github.com/jomali}}\\
\end{minipage}

\vspace{0.5cm}

%
%	INTRODUCTION, SKILLS AND TECHNOLOGIES
%

\cvsect{Presentación}

\begin{minipage}[t]{0.45\textwidth} % 40% of the page width for the introduction text
	\vspace{-\baselineskip} % Required for vertically aligning minipages
	
	Graduado en Ingeniería Informática por la Universidad de Cantabria. Durante mi etapa académica me interesé en el estudio de la desigualdad social observable en el seno de las TIC y la búsqueda de formas de acercar los computadores a personas con perfiles poco técnicos y/o con barreras de accesibilidad. Inclinaciones que han conducido mi carrera profesional hacia el desarrollo \emph{front-end}, con una preocupación especial en la experiencia de usuario.\\
\end{minipage}
\hfill % Whitespace between
\begin{minipage}[t]{0.5\textwidth} % 50% of the page for the skills bar chart
	\vspace{-\baselineskip} % Required for vertically aligning minipages
	\begin{barchart}{5.5}
		\baritem{JavaScript}{100}
		\baritem{React JS}{100}
		\baritem{Git}{75}
		\baritem{MySQL}{60}
		\baritem{Java}{60}
		\baritem{Spring}{35}
	\end{barchart}
\end{minipage}

% \begin{center}
% 	\bubbles{5/Eclipse, 6/git, 4/Office, 3/Inkscape, 3/Blender}
% \end{center}

%
%	EXPERIENCE
%

\cvsect{Experiencia}

\begin{entrylist}
	\entry
		{8/2022 --\\ Actualidad}
		{Desarrollador \emph{front-end}}
		{Indra Producción Software S.L.}
		{Integrado en equipo de desarrollo de un producto de gestión interna para \emph{Inditex}. Encargado de la redefinición y refactorización de partes de la aplicación para corregir deuda técnica heredada de etapas iniciales del software. Participo además en el desarrollo de nuevas funcionalidades.\\ \texttt{JS}\slashsep\texttt{React}}
	\entry
		{12/2019 -- 7/2022}
		{Desarrollador \emph{front-end}}
		{SOINCON Soluciones Industriales de Conectividad S.L.}
		{Desarrollador de \emph{front-end} en el software para gestión de RR.HH. \emph{Digital People}. He asumido también tareas de responsable de producto, ayudando a coordinar al resto del equipo de desarrollo y participando en la definición del \emph{roadmap} junto al equipo de dirección. He ofrecido apoyo adicional en la gestión con cliente; encargándome de tareas de comunicación y soporte, toma de requisitos funcionales y formaciones. Desarrollo \emph{full stack} puntual en otros proyectos de menor envergadura.\\ \texttt{JS}\slashsep\texttt{React}\slashsep\texttt{Material UI}\slashsep\texttt{Java}\slashsep\texttt{Spring}}
	\entry
		{10/2016 -- 5/2017}
		{Desarrollador \emph{front-end}}
		{CIC Consulting Informático de Cantabria S.L.}
		{Trabajo con representación gráfica de datos. Formación interna en el desarrollo \emph{full stack} de aplicaciones web sobre ecosistemas Java.\\ \texttt{JS}\slashsep\texttt{D3.JS}\slashsep\texttt{Java}\slashsep\texttt{Spring}\slashsep\texttt{Vaadin}}
	\entry
		{3/2015 -- 9/2015}
		{Prácticas como desarrollador web}
		{La Refactoría S.L.}
		{Incluyendo desarrollos a medida sobre la plataforma de compras por Internet \emph{PrestaShop}.\\ \texttt{PHP}\slashsep\texttt{JS}}
\end{entrylist}

%
%	EDUCATION
%

\cvsect{Educación}

\begin{entrylist}
	\entry
		{2010 -- 2019}
		{Grado en Ingeniería Informática}
		{Facultad de Ciencias. Universidad de Cantabria}
		{Mención en Computación: computación teórica, algoritmos, lenguajes formales y gráficos por computador. Trabajo de fin de grado centrado en la inclusión social a través del entretenimiento electrónico titulado: \emph{Sistema software de ficción interactiva como fórmula de inclusión digital}.}
\end{entrylist}

%
%	ADDITIONAL INFORMATION
%

\begin{minipage}[t]{0.35\textwidth}
	\vspace{-\baselineskip} % Required for vertically aligning minipages

	\cvsect{Idiomas}
	
	\textbf{Español} -- nativo\\
	\textbf{Inglés} -- usuario independiente (nivel B2 acreditado a través de la Capacitación Lingüística de la Universidad de Cantabria)\\
\end{minipage}
\hfill
\begin{minipage}[t]{0.6\textwidth}
	\vspace{-\baselineskip} % Required for vertically aligning minipages
	
	\cvsect{Aficiones y reconocimientos}

	He creado algunos videojuegos de texto accesibles que han sido seleccionados para exposiciones como \emph{Ficción interactiva y juegos narrativos} de la Universidad de Granada (marzo de $2019$) o la muestra \emph{I Jam Cultura Abierta} organizada por la asociación DEV (Madrid, enero de $2019$). Fui miembro fundador de una asociación cultural: \emph{"Agrupación Escénica Unos Cuantos"} ($2012$). Últimamente estudio asignaturas de un Grado en Filosofía en mis ratos libres.
\end{minipage}

\end{document}

%%%%%%%%%%%%%%%%%%%%%%%%%%%%%%%%%%%%%%%%%
% Developer CV
% LaTeX Template
% Version 1.0 (28/1/19)
%
% This template originates from:
% http://www.LaTeXTemplates.com
%
% Authors:
% Jan Vorisek (jan@vorisek.me)
% Based on a template by Jan Küster (info@jankuester.com)
% Modified for LaTeX Templates by Vel (vel@LaTeXTemplates.com)
%
% License:
% The MIT License (see included LICENSE file)
%
%%%%%%%%%%%%%%%%%%%%%%%%%%%%%%%%%%%%%%%%%

%
%	PACKAGES AND OTHER DOCUMENT CONFIGURATIONS
%

\documentclass[9pt]{developercv} % Default font size, values from 8-12pt are recommended

\begin{document}

%
%	TITLE AND CONTACT INFORMATION
%

\begin{minipage}[t]{0.6\textwidth}
	\vspace{-\baselineskip} % Required for vertically aligning minipages

	\colorbox{black}{{\HUGE\textcolor{white}{\textbf{\MakeUppercase{Martín Lisaso,}}}}}
	\colorbox{black}{{\HUGE\textcolor{white}{\textbf{José Francisco}}}}

	\vspace{6pt}

	{\huge Desarrollador \emph{front-end}}
\end{minipage}
\hfill
\begin{minipage}[t]{0.3\textwidth}
	\vspace{-\baselineskip} % Required for vertically aligning minipages

	\icon{MapMarker}{12}{Santander, Cantabria}\\
	\icon{Phone}{12}{+34 651 667 661}\\
	\icon{At}{12}{\href{mailto:jfm.lisaso@gmx.com}{jfm.lisaso@gmx.com}}\\
	\icon{Globe}{12}{\href{https://jomali.itch.io/}{jomali.itch.io}}\\
	% \icon{Github}{12}{\href{https://github.com/jomali}{github.com/jomali}}\\
\end{minipage}

\vspace{0.5cm}

%
%	INTRODUCTION, SKILLS AND TECHNOLOGIES
%

\cvsect{Presentación}

\begin{minipage}[t]{0.45\textwidth} % 40% of the page width for the introduction text
	\vspace{-\baselineskip} % Required for vertically aligning minipages
	
	Graduado en Ingeniería Informática por la Universidad de Cantabria con más de tres años de experiencia desarrollando aplicaciones web. Durante mi etapa académica me interesé en el estudio de la desigualdad social observable en el seno de las TIC y la búsqueda de formas de acercar los computadores a personas con perfiles poco técnicos y/o con barreras de accesibilidad. Estas inclinaciones han conducido mi carrera profesional hacia el desarrollo \emph{front-end}, con una preocupación especial en la experiencia de usuario.\\
\end{minipage}
\hfill % Whitespace between
\begin{minipage}[t]{0.5\textwidth} % 50% of the page for the skills bar chart
	\vspace{-\baselineskip} % Required for vertically aligning minipages
	\begin{barchart}{5.5}
		\baritem{JavaScript}{100}
		\baritem{React JS}{100}
		\baritem{React Native}{40}
		\baritem{Java}{70}
		\baritem{Spring}{60}
		\baritem{MySQL}{60}
		\baritem{Git}{60}
	\end{barchart}
\end{minipage}

% \begin{center}
% 	\bubbles{5/Eclipse, 6/git, 4/Office, 3/Inkscape, 3/Blender}
% \end{center}

%
%	EXPERIENCE
%

\cvsect{Experiencia}

\begin{entrylist}
	\entry
		{12/2019 --\\ Actualidad}
		{Desarrollador \emph{front-end}}
		{SOINCON Soluciones Industriales de Conectividad S.L.}
		{Responsable del aspecto \emph{front-end} en el producto para gestión de RR.HH. \emph{Digital People}. Ofrezco apoyo adicional en el desarrollo \emph{back-end}, así como en la gestión con cliente; asumiendo tareas de comunicación y soporte, formaciones y toma de requisitos funcionales. He llevado a cabo el desarrollo \emph{full stack} en otros proyectos de menor envergadura.\\ \texttt{JS}\slashsep\texttt{React}\slashsep\texttt{Java}\slashsep\texttt{Spring}}
	\entry
		{10/2016 -- 5/2017}
		{Desarrollador \emph{full stack}}
		{CIC Consulting Informático de Cantabria S.L.}
		{Trabajo con representación gráfica de datos. Formación interna en el desarrollo \emph{full stack} de aplicaciones web sobre ecosistemas Java.\\ \texttt{JS}\slashsep\texttt{D3.JS}\slashsep\texttt{Java}\slashsep\texttt{Spring}\slashsep\texttt{Vaadin}}
	\entry
		{3/2015 -- 9/2015}
		{Desarrollador \emph{full stack}}
		{La Refactoría S.L.}
		{Desarrollo web en proyectos de corto alcance, incluyendo desarrollos a medida sobre la plataforma de compras por Internet \emph{PrestaShop}.\\ \texttt{PHP}\slashsep\texttt{JS}}
\end{entrylist}

%
%	EDUCATION
%

\cvsect{Educación}

\begin{entrylist}
	\entry
		{2010 -- 2019}
		{Grado en Ingeniería Informática}
		{Facultad de Ciencias. Universidad de Cantabria}
		{Mención en Computación: computación teórica, algoritmos, lenguajes formales y gráficos por computador. Trabajo de fin de grado centrado en la inclusión social a través del entretenimiento electrónico titulado: \emph{Sistema software de ficción interactiva como fórmula de inclusión digital}.}
\end{entrylist}

%
%	ADDITIONAL INFORMATION
%

\begin{minipage}[t]{0.3\textwidth}
	\vspace{-\baselineskip} % Required for vertically aligning minipages

	\cvsect{Idiomas}
	
	\textbf{Español} -- nativo\\
	\textbf{Inglés} -- usuario independiente (nivel B2 acreditado a través de la Capacitación Lingüística de la Universidad de Cantabria)\\
\end{minipage}
\hfill
\begin{minipage}[t]{0.3\textwidth}
	\vspace{-\baselineskip} % Required for vertically aligning minipages
	
	\cvsect{Aficiones}

	Estudio asignaturas de un Grado en Filosofía en mis ratos libres. He participado también en varios colectivos dramáticos como el grupo \emph{"Entrecajas Fusión Teatro"} (2015-2018), la \emph{"Agrupación Escénica Unos Cuantos"} (2012-2015) --asociación cultural de la que fui miembro fundador junto a otros compañeros-- o el taller de teatro de la Universidad de Cantabria (2007-2011). Una experiencia que me ha ayudado a desarrollar mis habilidades comunicativas.

	% Fui miembro fundador junto a varios compañeros de la asociación cultural \emph{"Agrupación Escénica Unos Cuantos"} en 2012. He participado como actor, además, en otros colectivos dramáticos como el taller de teatro de la Universidad de Cantabria (2007-2011) o el grupo \emph{"Entrecajas Fusión Teatro"} (2015-2018). Una experiencia que me ha ayudado a desarrollar mis habilidades comunicativas.
\end{minipage}
\hfill
\begin{minipage}[t]{0.3\textwidth}
	\vspace{-\baselineskip} % Required for vertically aligning minipages
	
	\cvsect{Reconocimientos}

	He creado algunos videojuegos de texto accesibles, con los que he conseguido los siguientes reconocimientos: 2º premio en \emph{Hack2Progress} (V ed.) organizada por la Universidad de Cantabria y CIC (noviembre de 2019); selección para participar en las exposiciones \emph{Ficción interactiva y juegos narrativos} de la Universidad de Granada (marzo de 2019), y la muestra \emph{I Jam Cultura Abierta} organizada por la asociación DEV (Madrid, enero de 2019).
\end{minipage}

\end{document}
